% Die Kopf- und Fußzeile werden mit Hilfe des Paketes scrpage2 konfiguriert.
% Durch die Verwendung von Innen- und Außenseite eignet sich das Layout für ein- und doppelseitige Dokumente.

% Latex unterscheidet zwischen dem normalen Seitenstil und dem Stil plain.
% Letzterer wird für den Beginn von Kapiteln verwendet.
% In diesem Dokument heißen die Stile scrheadings und plain.scrheadings, da das Paket scrlayer-scrpage verwendet wird.
% LaTeX schaltet automatisch zwischen beiden um, wenn einer aktiviert wurde (mit \pagestyle{scrheadings}).

% Die folgenden Befehle benötigen deshalb 2 Parameter.
% Der optionale Parameter (in eckigen Klammern) ist für den Seitenstil plain.scrheadings gedacht.
% Der zweite Parameter (in geschweiften Klammern) ist für den Seitenstil scrheadings gedacht.
% Beginnt ein neues Kapitel, dann ist es nicht nötig, dass man das Kapitel erneut in die Kopfzeile schreibt.
% Auf allen anderen Seiten soll die Titelseite den Autor (Links) und das Kapitel (Rechts) enthalten.
% Die Fußzeile soll bei allen Seiten identisch sein.

% Innenseite der Kopfzeile
\ihead[]{\varAutor}
% Mitte der Kopfzeile
\chead{}
% Außenseite der Kopfzeile
\ohead[]{\headmark}

% Innnenseite der Fußzeile
\ifoot[]{}
% Mitte der Fußzeile
\cfoot[]{}
% Aussenseite der Fußzeile
\ofoot[\pagemark]{\pagemark}

% Alternative, wie sie häufig bei Büchern verwendet wird:
%\ihead[]{}
%\chead{}
%\ohead[\pagemark]{\headmark\qquad\pagemark}
%\ifoot[]{}
%\cfoot[]{}
%\ofoot[]{}
