% Der Befehl \newcommand kann auch benutzt werden um Variablen zu definieren:

% Titel der Arbeit:
    \newcommand{\varTitel}{Vorlage für eine Abschlussarbeit}
% Datum der Abgabe:
    \newcommand{\varDate}{\today}
% Autoren der Arbeit:
    \newcommand{\varAutor}{Michael Entrup}
% Weitere angaben unterhalb des Autors:
	\newcommand{\varInfo}{Abschlussarbeit\\im Fachbereich Physik\\der WWU Münster}
% E-Mail-Adressen der Autoren:
% Enthält eure Adresse einen Unterstrich, so müsst ihr '\_' verwenden!
    \newcommand{\varEmail}{michael.entrup@wwu.de}
% E-Mail-Adresse anzeigen (true/false):
    \newcommand{\varZeigeEmail}{true}
% Literaturverzeichnis anzeigen (true/false):
    \newcommand{\varZeigeLiteraturverzeichnis}{true}
% Stil der Einträge im Literaturverzeichnis
    \newcommand{\varLiteraturLayout}{unsrtdin}
